% Include Preamble 
\include{preamble}
\begin{document}
\maketitle
% Sectuion Title 
\section{Section--II: Basic Operations in LINUX	}
% Slide-0
\begin{frame}[t]{Basics}
	\begin{itemize}
		\item  Remember the UNIX/LINUX command line is case sensitive!
		\item The hash (pound) sign \# indicates end of a command and 
		the start of a comment.
		\item The notation \textless \dots \textgreater refers to variables 
		and file names that need to be specified by the user. The symbols 
		\textless  and \textgreater need to be excluded.
	\end{itemize}
\end{frame}


% Slide-0
\begin{frame}[t]{Navigation}\hfill
\begin{center}
	\begin{tabular}{lll} 
		\textbf{Command} & \textbf{Function} & \textbf{Syntax} \\ 
		\hline
		pwd & Present working directory & \color{red} \textdollar pwd \\ 
		ls & List contents of pwd &  \color{red} \textdollar ls  \\ 
		ls -l & Similar as ls, but provides additional info &  \color{red} 
		\textdollar ls -l \\ 
		ls -la & Includes hidden files (.name) as well &  \color{red} 
		\textdollar ls -la \\ 
	
	\end{tabular}
\end{center}
\end{frame}



% Slide-1
\begin{frame}[t]{Directory Operations}\hfill
	\begin{center}
		\begin{tabular}{lll} 
			\textbf{Command} & \textbf{Function} & \textbf{Syntax} \\ 
			\hline
			mkdir & Creates specified directory & \color{red} \textdollar mkdir 
			dir\textunderscore name\\ 
			cp &Copy file/directory as specified in path (-r to include content 
			in directories) & \color{red} \textdollar mkdir 
			dir\textunderscore name\\ 
			
			
			
			
			rmdir & Removes empty directory & \color{red}  \textdollar rmdir 
			dir\textunderscore name	\\  
			rmdir -r & Removes directory including its content & \color{red} 
			\textdollar rm 
			-r 	dir\textunderscore name	\\  
			rmdir -rf & Removes directory including its content & \color{red} 
			\textdollar rm -rf 	dir\textunderscore name
		\end{tabular}
	\end{center}
\end{frame}



% Slide-2
\begin{frame}[t]{File Operations}\hfill
	\begin{center}
		\begin{tabular}{lll} 
			\textbf{Command} & \textbf{Function} & \textbf{Syntax} \\ 
			\hline
			touch & Creates specified file & \color{red} \textdollar touch 
			file\textunderscore name\\ 
			rm & Removes empty file & \color{red}  \textdollar rm 
			file\textunderscore name	\\  
			rm -r & Removes file including its content & \color{red} 
			\textdollar rm -r 	file\textunderscore name	\\  
			rmdir -rf & Removes directory including its content & \color{red} 
			\textdollar rm -rf 	file\textunderscore name
		\end{tabular}
	\end{center}
\end{frame}



% Slide-3
\begin{frame}[t]{File Manipulations}\hfill
	\begin{center}
		\begin{tabular}{lll} 
			\textbf{Command} & \textbf{Function} & \textbf{Syntax} \\ 
			\hline
			touch & Create files & touch \color{red} filenmae.extension \\ 
			pwd & Present working directory & pwd \\ 
			cd & Change directory & \color{red}cd [Directory] \\ 
			cd .. & Previous Directory & \color{red} cd ..
		\end{tabular}
	\end{center}
\end{frame}



% Slide-4
\begin{frame}[t]{Compression}\hfill
	\begin{center}
		\begin{tabular}{lll} 
			\textbf{Command} & \textbf{Function} & \textbf{Syntax} \\ 
			\hline
			touch & Create files & touch \color{red} filenmae.extension \\ 
			pwd & Present working directory & pwd \\ 
			cd & Change directory & \color{red}cd [Directory] \\ 
			cd .. & Previous Directory & \color{red} cd ..
		\end{tabular}
	\end{center}
\end{frame}






% Slide-5
\begin{frame}[t]{Shortcuts}\hfill
	\begin{center}
		\begin{tabular}{lll} 
			\textbf{Command} & \textbf{Function} & \textbf{Syntax} \\ 
			\hline
			touch & Create files & touch \color{red} filenmae.extension \\ 
			pwd & Present working directory & pwd \\ 
			cd & Change directory & \color{red}cd [Directory] \\ 
			cd .. & Previous Directory & \color{red} cd ..
		\end{tabular}
	\end{center}
\end{frame}




% Slide-end
\begin{frame}[t]{References}
	
\end{frame}

% Thank you slide 
\plain{Thank You\\ \ \\ \Huge{\smiley}}

\end{document}